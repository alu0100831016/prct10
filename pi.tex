\documentclass[a4paper,10pt]{article}
%\documentclass[a4paper,10pt]{scrartcl}

\usepackage[utf8]{inputenc}
\usepackage{graphi cx}



\begin{document}
\title{Aproximacion del numero $\pi$}
\author{Alba De León Hernández\\Prctica de laboratorio \#10}
\date{9 de abril de 2014}
\maketitle
%--------------

\begin{abstract}
 Este es el resumen de la practica.
\end{abstract}
%--------------

\section{Seccion 1}
Esto es la seccion 1.
En la bibliografia \ref{} vemos...
\subsection{Subseccion 1}
Esto es la subseccion 1.
\footnote{Esto es el pie de pagina}
\subsection{Subseccion 2}
Esto es la subseccion 2.
\begin{tabular}{lrc}
Nombre&Edad&Clase \\
\hline
Alba & 19 & D \\
Elizabeth & 20 & B \\
Aidé & 21 & C \\
\label{table}
\end{tabular}
%-------------------

\section{Seccion 2}
Esto es la seccion 2.
\cite{pi}
\cite{pi2}
\begin{figure}
\includegraphics[scale=0.10]{imagen1.eps}
\label{fig1}
\end{figure}
\subsection{Subseccion 1}
Esto es la subseccion 1.
En la grfica \ref{fig1}
\subsection{Subseccion 2}
Esto es la subseccion 2.
En la tabla \ref{table} vemos...

\begin{thebibliography}{1}
\bibitem{pi}  http://es.wikipedia.org/
\bibitem{pi2} http://es.wikipedia.org/wiki/Pi
\end{thebibliography}


\end{document}

